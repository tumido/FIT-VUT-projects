\documentclass{beamer}
\usepackage[czech]{babel}
\usepackage[utf8x]{inputenc}
\usetheme{Malmoe}


\title{Typografie a~publikování}
\subtitle{SAPO\,--\,Samočinný počítač}
\author{Tomáš Coufal}
\institute{Fakulta informačních technologií \\ Vysoké učení technické v~Brně}
\date{\today}

\begin{document}
    \frame{\titlepage}
    \begin{frame}
       \frametitle{Osnova}
        \tableofcontents
    \end{frame}
\section{SAPO\,--\,po stránce historické}
\subsection{Historické souvislosti}
    \begin{frame}
        \frametitle{Historické souvislosti}
        \begin{itemize}
            \item Doba 50.\,lét 20.\,století, politické situace v ČSSR
            \item Kybernetika zakázána Stalinem
            \item Neexistujicí pracoviště
            \begin{itemize}
                \item ČVUT
                \item Nutnost vlastního oddělení
            \end{itemize}
        \end{itemize}
    \end{frame}
\subsection{Vznik SAPO}
    \begin{frame}
        \frametitle{Vznik SAPO}
        \framesubtitle{rok 1957}
        \begin{itemize}
            \item Projekt byl hotov roku 1951
            \item Stavba trvala 6 let
            \begin{itemize}
              \item Politické obstrukce
              \item Národní podnik Aritma
            \end{itemize}
            \item 2 roky ostrého provozu
        \end{itemize}
    \end{frame}
\subsection{Následné události}
    \begin{frame}
        \frametitle{Následné události}
        \begin{itemize}
            \item Následník EPOS
            \item Emigrace prof. Svobody
        \end{itemize}
    \end{frame}
\section{SAPO\,--\,po stránce technické}
\subsection{Vývoj}
    \begin{frame}
        \frametitle{Úvod}
        \begin{itemize}
            \item Vytvořen pod vedením prof. Svobody
            \item SAPO přináší na svou dobu revoluční prvky
            \item V západním bloku na spoustu technických vymožeností přichází až o mnoho let pozdeji
        \end{itemize}
    \end{frame}
\subsection{Specifika a rarity}
    \begin{frame}
        \frametitle{Specifika a rarity}
        \begin{itemize}
            \item První chybám odolný počítač na světě
            \item Místo elektronek použita relé
            \begin{itemize}
                \item Z důvodů výrobních kapacit (použito 7 tisíc relé)
                \item Nespolehlivost
            \end{itemize}
            \item Výkon 10 tisíc operací za hodinu
            \begin{itemize}
                \item 3 operace za vteřinu
                \item S 32bitovou přesností
                \item Možností výpočtů v plovoucí desetinné čárce
            \end{itemize}
        \end{itemize}
    \end{frame}
\subsection{Nové technologie}
    \begin{frame}
        \frametitle{Nové technologie}
        \begin{itemize}
            \item Tři aritmeticko-logické jednotky
                \begin{itemize}
                    \item Kvůli nespolehlivosti relé
                    \item Z tohoto důvodu odolnost vůči chybám
                \end{itemize}
            \item Duplicitní zápis dat
        \end{itemize}
    \end{frame}
\section{Závěr}
\subsection{Zdroje}
    \begin{frame}
        \frametitle{Děkuji za pozornost}
        \begin{itemize}
            \item
        \end{itemize}
    \end{frame}
\end{document}
