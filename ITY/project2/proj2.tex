\documentclass[a4paper, 11pt]{article}
\usepackage[czech]{babel}
\usepackage[utf8x]{inputenc}
\usepackage[IL2]{fontenc}
\usepackage{times}
\usepackage[left=1.5cm, top=2.5cm, text={18cm, 25cm}]{geometry}
\usepackage[unicode]{hyperref}

\usepackage{amsmath}
\usepackage{amsthm}
\usepackage{enumitem}
\usepackage{amssymb}

\theoremstyle{plain}
\newtheorem{noreset}{DEFAULT}[section]

\theoremstyle{definition}
\newtheorem{definice}[noreset]{Definice}
\theoremstyle{plain}
\newtheorem{algoritmus}[noreset]{Algoritmus}
\newtheorem{veta}{Věta}

\begin{document}
\thispagestyle{empty}
\begin{center}
  {\Huge\textsc{Fakulta informačních technologií}} \\
  {\Huge\textsc{Vysoké učení technické v~Brně}} \\
  \vspace{\stretch{0.382}}
  {\LARGE{Typografie a publikování\,--\,2. projekt}} \\
  {\LARGE{Sazba dokumentů a matematických výrazů}}
  \vspace{\stretch{0.618}}
\end{center}
{\Large{2015 \hfill Tomáš Coufal}}
\cleardoublepage
\setcounter{page}{1}
\twocolumn
\section*{Úvod}\label{strana:1}

V~této úloze si vyzkoušíme sazbu titulní strany, matematických vzorců, prostředí a dalších textových struktur obvyklých pro technicky zaměřené texty (například rovnice \eqref{rovnice:1} nebo definice \ref{definice:1} na straně \pageref{strana:1}).

Na titulní straně je využito sázení nadpisu podle optického středu s~využitím zlatého řezu. Tento postup byl probírán na přednášce.

\section{Matematický text}
Nejprve se podíváme na sázení matematických symbolů a výrazů v~plynulém textu. Pro množinu $V$ označuje $\text{card}(V)$ kardinalitu $V$.
Pro množinu $V$ reprezentuje $V^*$ volný monoid generovaný množinou $V$ s~operací konkatenace.
Prvek identity ve volném monoidu $V^*$ značíme symbolem $\varepsilon$.
Nechť $V^+ = V^* - \{ {\varepsilon} \}$ Algebraicky je tedy $V^+$ volná pologrupa generovaná množinou $V$ s~operací konkatenace.
Konečnou neprázdnou množinu $V$ nazvěme $abeceda$.
Pro $w \in V^*$ označuje $|w|$ délku řetězce $w$. Pro $W \subseteq V$ označuje $\text{occur}(w,W)$ počet výskytů symbolů z~$W$ v~řetězci $W$ a $\text{sym}(w,i)$ určuje $i$-tý symbol řetězce $w$; například $\text{sym}(abcd,3) = c$.

Nyní zkusíme sazbu definic a vět s~využitím balíku \verb|amsthm|.

\begin{definice}\label{definice:1}
\emph{Bezkontextová gramatika} je čtveřice $G = (V,T,P,S)$, kde $V$ je totální abeceda, $T \subseteq V$ je abeceda terminálů, $S \in (V-T)$ je startující symbol a $P$ je konečná množina \emph{pravidel} tvaru $q\!: A\to \alpha$, kde $A \in (V-T)$, $\alpha \in V^*$ a $q$ je návěští tohoto pravidla.
Nechť $N = V - T$ značí abecedu neterminálů.
Pokud $q\!: A\to \alpha \in P$, $\gamma$ , $\delta \in V^*$ provádí derivační krok z~${\gamma}A\delta$ do $\gamma\alpha\delta$ podle pravidla $q\!: A\to \alpha$, symbolicky píšeme ${\gamma}A\delta \Rightarrow  \gamma\alpha\delta\ [q\!: A \to \alpha]$ nebo zjednodušeně ${\gamma}A\delta \Rightarrow \gamma\alpha\delta$.
Standardním způsobem definujeme $\Rightarrow^m$, kde $m \geq 0$.
Dále definujeme tranzitivní uzávěr $\Rightarrow^+$ a tranzitivně-reflexivní uzávěr $\Rightarrow^*$.
\end{definice}

Algoritmus můžeme uvádět podobně jako definice textově, nebo využít pseudokódu vysázeného ve vhodném prostředí (například \verb|algorithm2e|).

\begin{algoritmus}
Algoritmus pro ověření bezkontextovosti gramatiky. Mějme gramatiku $G = (N, T, P, S)$.
\begin{enumerate}
  \item\label{bod:1} Pro každé pravidlo $p \in P$ proveď test, zda $p$ na levé straně obsahuje právě jeden symbol z~$N$.
  \item Pokud všechna pravidla splňují podmínku z~kroku \ref{bod:1}, tak je gramatika $G$ bezkontextová.
\end{enumerate}
\end{algoritmus}

\begin{definice}
\emph{Jazyk} definovaný gramatikou $G$ definujeme jako $L(G) = \{w \in T^* | S \Rightarrow^* w\}$.
\end{definice}

\subsection{Podsekce obsahující větu}

\begin{definice}
Nechť $L$ je libovolný jazyk. $L$ je \emph{bezkontextový jazyk}, když a jen když $L = L(G)$, kde $G$ je libovolná bezkontextová gramatika.
\end{definice}

\begin{definice}
% tady jsem nevedel jak jinak zaridit, aby doslo ke zalomeni radku
% (pokud bych v $$ cast pouzil \text{ je bezkontextovy jazyk}, pak se to nezalomi)
% takze jsem radsi $ usek prerusil a vysazel text normalne a pak vlozil zase matematickou zavorku
Množinu $\mathcal{L}_{CF} = \{L|L$ je bezkontextový jazyk$\}$ nazýváme \emph{třídou bezkontextových jazyků}.
\end{definice}

\begin{veta}\label{veta:1}
Nechť $L_{abc} = \{a^nb^nc^n|n \geq 0\}$ Platí, že $L_{abc} \notin \mathcal{L}_{CF}$.
\end{veta}

\begin{proof}
  Důkaz se provede pomocí Pumping lemma pro bezkontextové jazyky, kdy ukážeme, že není možné, aby platilo, což bude implikovat pravdivost věty \ref{veta:1}.
\end{proof}

\section{Rovnice a odkazy}

Složitější matematické formulace sázíme mimo plynulý text. Lze umístit několik výrazů na jeden řádek, ale pak je třeba tyto vhodně oddělit, například příkazem \verb|\quad|.

$$\sqrt[x^2]{y_0^3} \quad \mathbb{N} = \{0,1,2,\ldots\} \quad x^{y^y} \neq x^{yy} \quad z_{i_j} \not\equiv z_{ij}$$

V~rovnici \eqref{rovnice:1} jsou využity tři typy závorek s~různou explicitně definovanou velikostí.

\begin{eqnarray}\label{rovnice:1} %puvodne jsem chtel pouzit \begin{equation} a {equation*}, ale tam by neslo zarovnat rovna se nad sebe
\bigg\{ \Big[ \big(a+b\big)*c \Big]^d+1\bigg\} & = & x \\
\nonumber \lim_{x \to \infty} \frac{\sin^2 x + \cos^2 x}{4} & = & y
\end{eqnarray}

V~této větě vidíme, jak vypadá implicitní vysázení limity $\lim_{x \to \infty} f(n)$ v~normálním odstavci textu. Podobně je to i s~dalšími symboly jako $\sum _{i}^n$ či $\bigcup_{A \in \mathcal{B}}$. V~případě vzorce $\lim\limits_{x \to 0} \frac{\sin x}{x} = 1$ jsme si vynutili méně úspornou sazbu příkazem \verb|\limits|.

\begin{eqnarray}
\int\limits_a^b f(x)\,\mathrm{d}x & = & - \int_b^a f(x)\,\mathrm{d}x \\
\left(\sqrt[5]{x^4} \right)' = \left(x^{\frac{4}{5}}\right)' & = & \frac{4}{5}x^{-\frac{1}{5}} = \frac{4}{5\sqrt[5]{x}} \\
\overline{\overline{A \vee B}} & = & \overline{\overline{A} \wedge \overline{B}}
\end{eqnarray}

\section{Matice}

Pro sázení matic se velmi často používá prostředí \verb|array| a závorky (\verb|\left|, \verb|\right|).

\begin{center}
$$
\left( \begin{array}{cc}
  a+b & b-a \\
  \widehat{\xi+\omega} & \hat{\pi} \\
  \vec{a} & \overleftarrow{AC} \\
  0 & \beta
\end{array} \right)
$$$$
A = \left \| \begin{array}{cccc}
  a_{11} & a_{12} & \ldots & a_{1n} \\
  a_{21} & a_{22} & \ldots & a_{2n} \\
  \vdots & \vdots & \ddots & \vdots \\
  a_{m1} & a_{m12} & \ldots & a_{mn}
\end{array} \right \|
$$$$
\left| \begin{array}{cc}
  t & u \\
  v & w
\end{array} \right|
= tw - uv
$$
\end{center}

Prostředí \verb|array| lze úspěšně využít i jinde.

$$
\binom{n}{k} = \left \{ \begin{array}{ll}
  \frac{n!}{k!(n-k)!} & \text{pro } 0 \leq k \leq n \\
  0 & \text{pro } k < 0 \text{ nebo } k-> n
\end{array} \right.
$$

\section{Závěrem}

V~případě, že budete potřebovat vyjádřit matematickou konstrukci nebo symbol a nebude se Vám dařit jej nalézt v~samotném {\LaTeX}u, doporučuji prostudovat možnosti balíku maker \AmS-\LaTeX.
Analogická poučka platí obecně pro jakoukoli konstrukci v~{\TeX}u.

\end{document}
